\documentclass{article}
\usepackage[top=3cm,bottom=3cm,left=3cm,right=3cm,marginparwidth=2.3cm,headsep=10pt,letterpaper]{geometry} % Page margins
\usepackage{graphicx}
\graphicspath{{imagenes/}} % Specifies the directory where pictures are stored
\usepackage[spanish]{babel}
\usepackage{microtype} % Slightly tweak font spacing for aesthetics
\usepackage[utf8]{inputenc} % Required for including letters with accents

\newcommand{\incluirimagen}[3]{
	\begin{figure}[h]\centering
		\includegraphics[width=0.6\textwidth]{#1}
		\caption{#2}
		\label{#3}
	\end{figure}
}
\title{Reporte Encuesta ANIP 2016}
\author{Varios autores}
\date{\today}

\begin{document}
\maketitle

\section{Introducción}

\section{Análisis}

[Aquí va algo sobre las preguntas que queremos responder.]

\incluirimagen{placeholder.jpg}{Este es el texto que apacerá debajo.}{fig:figura}

\subsection{Frecuencias}

El cuadro \ref{tab:count} presenta la cantidad de encuestados que cursan y/o poseen posgrado (filas), y que cursan y/o poseen un posdoctorado (columnas). La mayor parte de los encuestados no estaba (al momento de la encuesta) cursando un posgrado y poseía un posgrado. De estos, la mayor parte de ellos ni cursa ni posee un posdoctorado.

Los cuadros \ref{tab:countwomen} y \ref{tab:countmen} presentan la frecuencia de mujeres y hombres, respectivamente, que poseen un master (MsC), un doctorado (PhD) y/o un posdoctorado. Los números entre paréntesis en el cuadro \ref{tab:countwomen} corresponden al porcentaje sobre el total de mujeres; ocurre de manera similar en el cuadro \ref{tab:countmen}.

% latex table generated in R 3.4.0 by xtable 1.8-2 package
% Tue Oct 24 15:03:25 2017
\begin{table}[ht]
\centering
\begin{tabular}{|p{3cm}|r|r|r|r|}
  \hline
 & No cursa postdoc, no tiene postdoc & No cursa postdoc, tiene postdoc & Cursa postdoc, no tiene postdoc & Cursa postdoc, tiene postdoc \\ 
  \hline
No cursa posgrado, no tiene posgrado & 0 & 3 & 0 & 0 \\ 
  No cursa posgrado, tiene posgrado & 247 & 144 & 118 & 48 \\ 
  Cursa posgrado, no tiene posgrado & 69 & 0 & 1 & 0 \\ 
  cursa posgrado, tiene posgrado & 110 & 0 & 3 & 0 \\ 
   \hline
\end{tabular}
\caption{Número de encuestados, según si poseen posgrado y/o postdocs.} 
\label{tab:count}
\end{table}

% latex table generated in R 3.4.0 by xtable 1.8-2 package
% Tue Oct 24 16:32:15 2017
\begin{table}[ht]
\centering
\begin{tabular}{rrrr}
  \hline
 & No tiene Postdoc & Tiene Postdoc & Subtotal \\ 
  \hline
No tiene MsC, No tiene PhD & 37 & 2 & 39 \\ 
  No tiene MsC, Tiene PhD & 91 & 42 & 133 \\ 
  Tiene MsC, No tiene PhD & 94 & 0 & 94 \\ 
  Tiene MsC, Tiene PhD & 61 & 17 & 78 \\ 
  Subtotal & 283 & 61 & 344 \\ 
   \hline
\end{tabular}
\caption{Número de mujeres con Msc y/o Phd.} 
\label{tab:countwomen}
\end{table}

% latex table generated in R 3.4.0 by xtable 1.8-2 package
% Tue Oct 24 16:32:15 2017
\begin{table}[ht]
\centering
\begin{tabular}{rrrr}
  \hline
 & No tiene Postdoc & Tiene Postdoc & Subtotal \\ 
  \hline
No tiene MsC, No tiene PhD & 33 & 1 & 34 \\ 
  No tiene MsC, Tiene PhD & 79 & 98 & 177 \\ 
  Tiene MsC, No tiene PhD & 89 & 1 & 90 \\ 
  Tiene MsC, Tiene PhD & 64 & 34 & 98 \\ 
  Subtotal & 265 & 134 & 399 \\ 
   \hline
\end{tabular}
\caption{Número de hombres con Msc y/o Phd.} 
\label{tab:counmen}
\end{table}


\subsection{¿Qué factores influyen sobre el que la persona tenga trabajo?}
% latex table generated in R 3.4.0 by xtable 1.8-2 package
% Tue Oct 24 15:09:27 2017
\begin{table}[ht]
\centering
\begin{tabular}{rrrrr}
  \hline
 & Estimate & Std. Error & z value & Pr($>$$|$z$|$) \\ 
  \hline
(Intercept) & -0.0401 & 0.7406 & -0.05 & 0.9568 \\ 
  bio.edad & 0.0181 & 0.0220 & 0.82 & 0.4103 \\ 
  bio.generoMasculino & -0.3991 & 0.2124 & -1.88 & 0.0603 \\ 
  aporte.numarticulos & 0.0982 & 0.0315 & 3.11 & 0.0019 \\ 
  aporte.numlibros & 0.0155 & 0.1558 & 0.10 & 0.9208 \\ 
  aporte.numproyectos.gana & 0.0897 & 0.0670 & 1.34 & 0.1803 \\ 
  sit.tiene.mscTRUE & -0.2375 & 0.2172 & -1.09 & 0.2742 \\ 
  sit.tiene.phdTRUE & 0.7501 & 0.2538 & 2.96 & 0.0031 \\ 
  sit.tiene.posTRUE & -0.3339 & 0.3028 & -1.10 & 0.2701 \\ 
   \hline
\end{tabular}
\caption{¿Qué factores influyen más en que la persona tenga trabajo?} 
\label{tab:reg1}
\end{table}


En el cuadro \ref{tab:reg1} se presenta una regresión logística realizada para responder la pregunta de qué factores influyen sobre el que las personas tengan trabajo. Las variables de entrada estudiadas son las siguientes:

\begin{itemize}
\item \textbf{bio.edad}: Corresponde a la edad reportada por cada persona.
\item \textbf{bio.genero (masculino)}: Corresponde al género de la persona. El análisis está hecho usando el género femenino como base de comparación.
\item \textbf{aporte.numarticulos}: Corresponde al número de artículos publicados por la persona.
\item \textbf{aporte.numlibros}: Corresponde al número de libros publicados por la persona.
\item \textbf{aporte.numproyectos.gana}: Corresponde al número de proyectos de investigación concursados que la persona ha ganado.
\item \textbf{sit.tiene.msc (TRUE)}: Corresponde a si la persona tiene o no un máster. La base de comparación es que la persona no lo tenga.
\item \textbf{sit.tiene.phd (TRUE)}: Corresponde a si la persona tiene o no un doctorado. La base de comparación es que la persona no lo tenga.
\item \textbf{sit.tiene.pos (TRUE)}: Corresponde a si la persona tiene o no un posdoctorado. La base de comparación es que la persona no lo tenga.
\end{itemize}

De las variables anteriores, tanto tener un doctorado ($e=0.75$, $p=0.0031$) como el número de artículos ($e=0.0982$, $p=0.0019$) influyen de manera significativa sobre el que la persona tenga trabajo. De éstas, la primera variable es la que tiene mayor influencia.

\subsection{Regresión 2}
% latex table generated in R 3.4.0 by xtable 1.8-2 package
% Tue Oct 24 16:32:15 2017
\begin{table}[ht]
\centering
\begin{tabular}{rrrrr}
  \hline
 & Estimate & Std. Error & t value & Pr($>$$|$t$|$) \\ 
  \hline
(Intercept) & 0.9603 & 0.2413 & 3.98 & 0.0001 \\ 
  bio.edad & 0.0030 & 0.0071 & 0.42 & 0.6766 \\ 
  bio.generoMasculino & -0.0794 & 0.0672 & -1.18 & 0.2378 \\ 
  aporte.numarticulos & 0.0001 & 0.0033 & 0.03 & 0.9795 \\ 
  aporte.numlibros & 0.0588 & 0.0416 & 1.41 & 0.1585 \\ 
  aporte.numproyectos.gana & -0.0047 & 0.0144 & -0.33 & 0.7445 \\ 
  sit.tiene.mscTRUE & -0.0497 & 0.0699 & -0.71 & 0.4773 \\ 
  sit.tiene.phdTRUE & 0.1210 & 0.0812 & 1.49 & 0.1363 \\ 
  sit.tiene.posTRUE & -0.0399 & 0.0861 & -0.46 & 0.6436 \\ 
   \hline
\end{tabular}
\caption{¿Qué factores influyen más en la cantidad de trabajos que tienen las personas?} 
\label{tab:reg2}
\end{table}


\subsection{Regresión 3}
% latex table generated in R 3.4.0 by xtable 1.8-2 package
% Tue Oct 24 17:48:05 2017
\begin{table}[ht]
\centering
\begin{tabular}{rrrrr}
  \hline
 & Estimate & Std. Error & z value & Pr($>$$|$z$|$) \\ 
  \hline
(Intercept) & -1.1321 & 0.7608 & -1.49 & 0.1368 \\ 
  bio.generoMasculino & 0.4369 & 0.1892 & 2.31 & 0.0209 \\ 
  bio.edad & 0.0503 & 0.0215 & 2.34 & 0.0191 \\ 
  sit.tiene.mscTRUE & 0.2191 & 0.1946 & 1.13 & 0.2603 \\ 
  sit.tiene.phdTRUE & -0.3335 & 0.2246 & -1.49 & 0.1375 \\ 
  sit.tiene.posTRUE & 0.9212 & 0.2532 & 3.64 & 0.0003 \\ 
  bio.essostenedorTRUE & 0.0785 & 0.1908 & 0.41 & 0.6807 \\ 
  bio.cuantoshijos & -0.0390 & 0.1112 & -0.35 & 0.7261 \\ 
  ins.anhelo.academicoTRUE & 0.0913 & 0.3432 & 0.27 & 0.7901 \\ 
   \hline
\end{tabular}
\caption{¿Qué factores influyen más en que la persona tenga un trabajo \emph{estable}?} 
\label{tab:reg3}
\end{table}


\subsection{Regresión 4}
% latex table generated in R 3.4.0 by xtable 1.8-2 package
% Tue Oct 24 17:48:05 2017
\begin{table}[ht]
\centering
\begin{tabular}{rrrrr}
  \hline
 & Estimate & Std. Error & t value & Pr($>$$|$t$|$) \\ 
  \hline
(Intercept) & 2.6990 & 0.6883 & 3.92 & 0.0001 \\ 
  bio.edad & -0.0154 & 0.0179 & -0.86 & 0.3916 \\ 
  bio.generoMasculino & 0.3220 & 0.1626 & 1.98 & 0.0482 \\ 
  bio.essostenedorTRUE & 0.5037 & 0.1712 & 2.94 & 0.0034 \\ 
  bio.cuantoshijos & 0.2307 & 0.0945 & 2.44 & 0.0150 \\ 
  ins.tienetrabajoTRUE & 1.8481 & 0.2033 & 9.09 & 0.0000 \\ 
  ins.anhelo.academicoTRUE & -0.3039 & 0.2996 & -1.01 & 0.3109 \\ 
   \hline
\end{tabular}
\caption{¿Explica el salario recibido la percepción de inserción? Este análisis se realiza para todo el universo de personas.} 
\label{tab:reg4}
\end{table}


\subsection{Regresión 5}
% latex table generated in R 3.4.0 by xtable 1.8-2 package
% Tue Oct 24 16:32:15 2017
\begin{table}[ht]
\centering
\begin{tabular}{rrrrr}
  \hline
 & Estimate & Std. Error & t value & Pr($>$$|$t$|$) \\ 
  \hline
(Intercept) & 2.4735 & 0.7908 & 3.13 & 0.0019 \\ 
  bio.edad & -0.0090 & 0.0198 & -0.45 & 0.6507 \\ 
  bio.generoMasculino & 0.1669 & 0.1776 & 0.94 & 0.3480 \\ 
  bio.essostenedorTRUE & 0.0660 & 0.1938 & 0.34 & 0.7338 \\ 
  bio.cuantoshijos & 0.1310 & 0.1008 & 1.30 & 0.1946 \\ 
  ins.anhelo.academicoTRUE & -0.2527 & 0.3257 & -0.78 & 0.4384 \\ 
  as.numeric(ins.ingreso) & 0.6222 & 0.0867 & 7.18 & 0.0000 \\ 
   \hline
\end{tabular}
\caption{¿Explica el salario recibido la percepción de inserción? Este análisis se realiza sólo para las personas que poseen trabajo.} 
\label{tab:reg5}
\end{table}


\end{document}
